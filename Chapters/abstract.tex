%%%%%%%%%%%%%%%%%% Abstract %%%%%%%%%%%%%%%%%%%%%%%%%%%%%%%%%%%%%%%%%%%%
\begin{abstract}
\begin{quotation}
\noindent Discrete exterior calculus (DEC) provides tools to discretize partial differential equations (PDEs) defined on manifolds.  The discretized PDE's can be described using global sparse linear equations defined on discrete manifolds. DEC has successfully been used in various computer graphics applications. Nevertheless, DEC is not directly accessible to many graduate students in computer sciences, as they can have varrying mathematical backgrounds, from basic to very advanced. DEC is directly related to exterior calculus (EC) and EC is often not covered in the math  covered standard computer science course.

This textbook aims to render DEC easily accessible to a broader public, by giving an introduction to DEC alongside of EC. Only a basic knowledge of calculus and linear algebra is assumed, covering even basic notions like manifolds. The geometric aspects of both DEC and EC are emphasized in order to put across the insights behind the DEC and EC formalisms. More advanced DEC and EC results are given in the context of applications, where their relevance is demonstrated at once. The use of DEC is demonstrated in the context of surface parametrization, vector field design, and fluid simulations. 


 This text provides a working knowledge of both exterior calculus and discrete exterior calculus, enabling the reader to apply and adapt DEC to new problems and to follow reasonings made using EC. 

%This thesis provides a working knowledge of both exterior calculus and discrete exterior calculus, enabling the reader to apply and adapt DEC to new problems and to follow reasonings made using EC. 
%
%\noindent Discrete exterior calculus (DEC) is a mathematical theory that provides tools to discretize differential equations defined on manifolds. DEC is directly defined on simplicial complexes and has successfully been used in various computer graphics applications. This thesis is an introduction to DEC, 
%accessible to any reader with a basic knowledge of multidimensional calculus. 
%
%We develop DEC alongside of exterior calculus (EC) %, rather than developing DEC as an independent theory. 
%%Like this DEC profits directly from insights in EC and vice versa. 
%and emphasize the geometric aspects of both theories in order to put across the insights behind their formalisms. More advanced DEC and EC results are given in the context of applications, where their relevance is demonstrated at once. The use of DEC is demonstrated in the context of surface parametrization, vector field design, and fluid simulations. 
%
%This thesis provides a working knowledge of both exterior calculus and discrete exterior calculus, enabling the reader to apply and adapt DEC to new problems and to follow reasonings made using EC. 
\end{quotation}
\end{abstract}