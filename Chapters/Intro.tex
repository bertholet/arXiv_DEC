\chapter{Introduction}
	
	\section{Goal of this Thesis}
	
	This thesis introduces the reader to exterior calculus (EC) and discrete exterior calculus (DEC). Exterior calculus provides a very elegant way to formulate important operators and relations from standard calculus and leads to deeper insights about various differential operators. Discrete exterior calculus mirrors exterior calculus, but is defined directly on triangle meshes and simplicial complexes. It allows to reformulate and approximate equations from exterior calculus with matrices, such that the equations can be solved computationally directly on the meshes. 
	
After reading this thesis a reader should have a working knowledge of both exterior calculus and discrete exterior calculus, such that you can apply discrete exterior calculus to problems and reason (or at least follow reasonings) using exterior calculus. 
	
Both EC and DEC are, under a layer of abstraction, of a very geometric nature. This thesis also attempts to help a reader see the simplicity and beauty of the underlying relations used by and described by EC and DEC.

\section{Why Use DEC?}

DEC provides one way to discretize partial differential equations, and naturally allows the treatment of differential equations on curved discrete surfaces or spaces, described by triangle meshes or more general simplicial complexes. The major DEC features are:
\begin{itemize}
	\item DEC is designed to preserve the geometry behind differential equations. The preservation of visually striking geometric features is important for computer graphic applications.
	\item DEC provides a set of coherently discretized differential operators like the gradient, the curl or the divergence. The relationship between these differential operators is preserved to a large extent by their DEC discretizations. For example Stokes' theorem and the Hodge decomposition theorem are preserved by DEC.
	\item For DEC there is no need to introduce local coordinates on meshes. The DEC operators are coordinate free.
	\item DEC is designed to be used on triangle mesh surfaces, tetrahedral meshes or more general simplicial complexes representing curved spaces. This is appealing for computer graphic applications, as triangle meshes and tetrahedral meshes are widely used.
	\item DEC discretizations naturally lead to sparse linear systems that can be solved efficiently by standard solvers.
\end{itemize}


\section{A Glimpse of DEC}
\label{sec:1_example}
\begin{figure}
\begin{center}
\includegraphics[height = 5cm]{imgs/1_1_SolvingProcessUsingDEC.eps}
\end{center}
\caption{Ideal solution process using DEC.}
\label{fig:1_solutionprocess}
\end{figure}
	
Broadly speaking, DEC provides a set of matrices that correspond to EC operators. EC operators on the other hand generalize standard calculus operators, like the divergence or gradient operator. 
The application of DEC to a problem usually follows the general pattern, sketched in Figure \ref{fig:1_solutionprocess}. At the beginning stands a problem statement or solution process expressed in standard calculus terms. In a first step the standard calculus equations are reformulated using exterior calculus operators. This is done in order to use the correspondence between EC operators and DEC matrices: in the exterior calculus formulation, the exterior calculus operators can be substituted directly by the corresponding discrete exterior calculus matrices. This produces linear equations that finally can be solved computationally, using standard sparse linear solvers. 

We illustrate this pattern with an application from Desbrun et al. \cite{Desbrun:1999:IFI:311535.311576}, \cite{laplacebeltrami}: the smoothing of two-dimensional meshes. This application can be based on the theory of minimal surfaces. In the setting of minimal surfaces a so called area minimizing flow, or mean curvature flow arises, see e.g. \cite{carmo1992differentialgeometrie} or \cite{2011arXiv1102.1411C}. For a curved surface, this flow is described by a set of vectors orthogonal to the surface. Then, if the whole surface is slightly moved in these directions, it is guaranteed that the area of the surface is decreased everywhere. Say $M$ is our surface and $V$ is the set of directions describing the mean curvature flow, then in a sloppy notation
\[\left. \frac{\partial}{\partial t} area(M + t\cdot V) \right|_{t=0}\leq 0.\]
Now it turns out that the mean curvature flow can be described by applying the Laplace operator $\Delta_M$ defined on the surface $M$, to the coordinates of the surface $M$, 
\[V = \begin{pmatrix}
\Delta_M M.x \\
\Delta_M M.y \\
\Delta_M M.z
\end{pmatrix}.\]
For the moment it is not important how a Laplacian is defined \emph{on} a surface, this will be described in Chapter \ref{chap:EC}. Important is only, that $\Delta$ is a standard calculus differential operator that can be expressed using exterior calculus operators, namely $d$ and $\partial$: 
\[\Delta = \partial d.\]
This gives us a description of the area minimizing process in exterior calculus terms; if the surface positions are changed overt time according to the following equation, its area gets minimized:
\[\frac{\partial}{\partial t} \begin{pmatrix}
M_t.x \\
M_t.y \\
 M_t.z
\end{pmatrix} = \begin{pmatrix}
\partial d \,M_t.x \\
\partial d\, M_t.y \\
\partial d\, M_t.z
\end{pmatrix}.\]
Now suppose we have a triangle mesh and we want to minimize its surface. Wanting to minimize the surface of a mesh is not an exotic idea. In this example the surface minimization is used to smooth a surface. If a surface is bumpy, minimizing the surface locally will flatten the bumps. DEC now provides the possibility to directly use the description of the area minimizing flow on the triangle mesh. The area minimizing process in described in exterior calculus terms. Discrete exterior calculus mirrors exterior calculus: it provides two matrices $\partial^{discrete}$ and $d^{discrete}$ that correspond to the EC operators $\partial$ and $d$. Then we simply replace the exterior calculus operator $\Delta = \partial d$ with the  discrete exterior calculus matrix $\Delta^{discrete} = \partial^{discrete} \cdot d^{discrete}$ and the surface coordinates of $M$ by the coordinates of the mesh vertices $v$ and we get
\[\frac{\partial}{\partial t} \begin{pmatrix}
v_t.x \\
v_t.y\\ 
v_t.z
\end{pmatrix} = \begin{pmatrix}
\Delta^{discrete} v_t.x \\
\Delta^{discrete} v_t.y \\
\Delta^{discrete} v_t.z
\end{pmatrix}.\]
This describes three equations, one for each coordinate, that need to be integrated. These equations can be integrated computationally, either using explicit integration,
\[v_{t_{n+1}}.x = v_{t_n}.x + (t_{n+1}-t_n) \Delta^{discrete} v_{t_n}.x,\]
or using implicit Euler integration,
\[v_{t_{n+1}}.x = v_{t_n}.x + (t_{n+1}-t_n) \Delta^{discrete} v_{t_{n+1}}.x,\]
which means solving the following sparse linear equation for $v_{t_{n+1}}.x $ in every time step, solving for the $x$-coordinates of all vertices at once
\[\underbrace{(Id - (t_{n+1}-t_n) \Delta^{discrete})}_{square\; matrix\; of \; dim \; \# vertices} v_{t_{n+1}}.x = v_{t_n}.x\;.\]
Implicit integration is much more stable than explicit integration and is the option that should be chosen. Lastly the mesh needs to be rescaled after every step, because it shrinks. For example it could be rescaled such that its volume stays constant. And then... this is it! This is all it takes to smooth surfaces as in Figure \ref{fig:1_dragonsmoothing}.  For implementation details see \cite{Desbrun:1999:IFI:311535.311576} and \cite{laplacebeltrami}; note that there the application is derived without the use of DEC.

\begin{figure}%
\includegraphics[width=\columnwidth]{imgs/1_dragon_smoothing.eps}%
\caption{Smoothing a dragon mesh, using the implicit integration scheme.}%
\label{fig:1_dragonsmoothing}%
\end{figure}

This is an ideal use-case of DEC and nothing is ever as simple as motivational examples make believe. The formulation with EC must be chosen carefully such that the translation to DEC works. In order to choose a good EC formulation a good knowledge of EC and its relation to DEC is vital. Often the DEC matrices need to be adapted before they can be used. %And this is why you need read this thesis; to understand the single steps and where they go wrong and why all this should work at all. %And you might wonder what the de Rham complex is. Or you might never have heard of it.


%		\begin{itemize}
%			\item Derivative as a simple example 
%			\item vs FEM 
%			\item show a picture of the deRahm complex saying that this diagram helps you translate Differential Operators into Matrices.
%		\end{itemize}
%				\begin{itemize}
%			\item From equations to Meshes.
%		\end{itemize}
%\pagebreak
\section{Structure of this Thesis}
The thesis has three pillars: smooth theory, discrete theory and applications/ implementations of example problems. As far as it is possible, the smooth theory is introduced alongside the corresponding discrete theory, such that the reasons and assumptions behind the discretizations become clear. The emphasis is to convey an intuitive understanding of the smooth theory rather than giving a rigorous mathematical introduction. 

Chapter \ref{chap:mfs_and_mshs} introduces manifolds and discrete manifolds, or more generally simplicial complexes. We cover orientations and border operators and sketch how manifolds allow calculus \emph{on} them. We also give some suggestions on how to implement discrete manifolds in a DEC setting.

In Chapter \ref{chap:diffforms} we explain differential forms and discrete differential forms. We introduce the $\wedge$ and $\sharp$ operators and tie  differential forms to standard calculus objects. We also consider integrals of differential forms and how discrete differential forms can be interpreted as sampled differential forms.

Chapter \ref{chap:EC} introduces the remaining exterior calculus operators, the exterior derivative $d$, the coderivative $\partial$ and the Hodge star $\star$. We look at Stokes' theorem, which explains the geometry of the exterior derivative and the standard calculus operators $curl$ and $div$ and focus on understanding the geometry of the Hodge star. These insights then are used for the definition of the DEC operators. 

The Chapters \ref{chap:meshparam}, \ref{chap:vfs} and \ref{chap:FS} then use EC and DEC for different computer graphics applications. We broaden the understanding of EC and DEC by explaining application dependent results and also shortly introduce application dependent theory.

In Chapter \ref{chap:meshparam} we consider conformal parameterizations of mesh patches with DEC. We mention Gortler's extension of Tutte's theorem and give simple boundary constraints to compute parametrizations. The computation of mesh parametrizations is for example useful to generate texture maps. 

In chapter \ref{chap:vfs} we use  DEC for vector field design. Being able to design vector fields can be a useful element for various applications, for example texture generation, as it is used in the original paper \cite{vField} or guided surface crack generation, as in \cite{kirtap}. In this context we present the Hodge decomposition and its discrete variant, results about harmonic forms and look at the Poincar\'e Hopf index theorem to highlight the influence of surface topology on solution spaces of problem statements.

Chapter \ref{chap:FS} describes a fluid simulation with DEC. Here DEC is used in a more complex setting; this application highlights best how problem statements need to be carefully reformulated for DEC to be applicable. We consider harmonic forms once more, and look at a result from de Rham which describes the degrees of freedom of harmonicity depending on the topology of the manifold, and how the degrees of freedom can be controlled on closed manifolds. 

\section{Disclaimer}
This thesis focuses on introducing readers without any prior experience with DEC and only a basic knowledge of calculus to DEC. But we omit many important subjects in this introduction to EC and DEC.  We do not consider any error analysis or compare DEC to finite element methods or any other alternatives. Furthermore, EC is not introduced rigorously, many technical details are omitted or only sketched. We refer to standard works for details, see also the discussion in Chapter \ref{chap:discussion}.
	