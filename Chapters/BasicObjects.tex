\chapter{Manifolds and Meshes}
\label{chap:mfs_and_mshs}

\begin{figure}[h]
\begin{center}
\includegraphics[height = 4cm]{imgs/2_1_ECVsDEC.eps}
\end{center}
\vspace{-0.5cm}
\caption{Exterior calculus is defined on manifolds, discrete exterior calculus on discrete manifolds. The continuous and the discrete theory are related via the geometric operations that manifolds and discrete manifolds have in common.  This chapter covers manifolds, discrete manifolds and geometric operations.}
\label{fig:2_ECVvsDEC}
\end{figure}


This chapter introduces the geometric objects exterior calculus (EC) and discrete exterior calculus (DEC) deal with. Everything treated in this thesis is inherently geometric, and geometric relations of objects lie at the heart of exterior calculus and discrete exterior calculus, as depicted in Figure \ref{fig:2_ECVvsDEC}. Geometric relations are meant in their broadest sense. How is an object oriented relatively to another? What is the geometric boundary of an object? How is a boundary oriented relatively to the interior of the object? The core realization in Chapter \ref{chap:EC} is that the most important differential operators are in some sense nothing else than boundary operators. 


This chapter covers the description of manifolds and some of their geometric features. Local coordinates, tangential spaces, orientations and the border operator are introduced. We also treat how directional derivatives can be computed \emph{on} manifolds, which allows differential calculus on manifolds. These topics form the Section \ref{sec::2_Manifolds} of this chapter.

Section \ref{sec::2_discreteManifolds} treats the discretization of manifolds: simplicial complexes. These allow the definition of orientations and border operators, thereby discretizing the geometric operations introduced on manifolds.

The last section of this chapter is of a more practical nature. It can be used as a guideline when implementing discrete manifolds for DEC. As application some geometric operations on meshes and discrete manifolds are described.

\section{Manifolds}
\label{sec::2_Manifolds}
We introduce manifolds as smooth curved $k$-dimensional subsets of $\mathbb R^n$. %Some simple two-dimensional manifolds, in short 2-manifolds, are depicted in Figure \ref{fig::2_1_manifold}.
In this section we cover the following topics on manifolds:
\begin{enumerate}
\item The description of manifolds using local maps
\item Tangential spaces
\item The orientation of manifolds
\item The border operator and the orientation of borders
\item Differential calculus on manifolds (derivatives of functions on manifolds)
\end{enumerate}
		
\subsection{Describing Manifolds}	

\begin{figure}
	\begin{center}
		\includegraphics[height=4cm]{imgs/2_1_2dmanifolds.eps}
	\end{center}
		\caption{Three simple 2-manifolds; the surfaces of the sphere and the torus do not have any borders, the third surface has. The volume contained by the sphere and the torus are 3-manifolds with border, their borders being the surfaces.}
		\label{fig::2_1_manifold}
\end{figure}

A $k$-dimensional manifold, or short $k$-manifold, is an object that locally looks like $\mathbb{R}^k$ and possibly has one or multiple $k-1$ dimensional boundaries. Some examples are given in Figure \ref{fig::2_1_manifold}.

Manifolds are described using local maps. Local maps describe small portions of manifolds. There are two types of maps: maps that describe manifolds around inner points and maps that describe manifolds around border points, as depicted in Figure \ref{fig::2_1_mapping}. %This is taken account of by allowing a local map to either describe the manifold using a plane or a halfplane. A 2-manifold then is an object that can be described at every point locally with a 2 dimensional map. 
Generally, a $k$-dimensional manifold $M$ lying in $\mathbb R^n$ is a geometric object that locally looks like $\mathbb R^k$ or, at its boundaries, like the halfspace $\mathbb H^k = \{x= (x_1,...,x_k) \in \mathbb R^k : x_k \geq 0\}$.  Formally this is guaranteed by requiring that for every point on the manifold there is a map linking some neighborhood of the point to $\mathbb R^k$ or $\mathbb H^k$.

For simplicity we assume henceforth that all functions and mappings are infinitely differentiable.
\[\]		
		
\begin{definition}[Map] A $k$-dimensional map or parametrization is a differentiable mapping 
\[\phi: U \rightarrow \phi(U)\subset \mathbb R^n,\] 
\[\begin{pmatrix}
	u_1\\ \vdots \\ u_k
\end{pmatrix} \rightarrow \begin{pmatrix}x_1(u_1,...,u_k)\\x_2(u_1,...,u_k)\\ \vdots \\ \vdots \\ x_n(u_1,...,u_k)\end{pmatrix},\]
that is injective and whose Jacobi matrix $D\phi$,
\[D\phi = \begin{pmatrix}
\frac{\partial\phi_1}{\partial u_1} & ... & \frac{\partial\phi_1}{\partial u_k} \\
\vdots & & \vdots \\
\frac{\partial\phi_n}{\partial u_1} & ... & \frac{\partial\phi_n}{\partial u_k}
\end{pmatrix},\]
 has rank $k$ on all $U$, where $U$ is some open subset of $\mathbb R^k$ or $\mathbb H^k$ (Figure \ref{fig::2_1_mapping}). 		
\end{definition} 

\begin{figure}
	\begin{center}
		\includegraphics[width=14cm]{imgs/2_1_borderedmanifold_maps.eps}
		\caption{An example of a bordered 2-manifold. Left: a local map $\phi$ at some inner point. Right: a local map at a border point. The map at the border has the additional restriction that the border of the halfspace $\mathbb H^2$ is mapped to the border of the manifold.}
		\label{fig::2_1_mapping}
	\end{center}
\end{figure}

There are some important details in the definition of maps. The injectivity of the maps prevents self intersecting manifolds, and the constraint on the rank of the Jacobi matrix makes sure that the image of a map $\phi$ does not degenerate. For example the image of a two dimensional map should not degenerate to a point or a line. A $k$-manifold then is an object where you can find local maps everywhere:

\begin{definition}[Manifold] A subset $S\subset \mathbb R^n$ is a $k$-manifold, if for each point $p \in S$ there exist an open set $V\subset \mathbb R^k$ such that there is a bijective map $\phi: U \rightarrow  \phi(U) = V\cap S$.
\end{definition} 
Note that in the condition $\phi(U) = V \cap S$ the additional constraint that the boundary of $\mathbb H^k$ is mapped to the boundary of $M$ is hidden. Also, technically, the surface patches in the Figures \ref{fig::2_1_manifold} and \ref{fig::2_1_mapping} that are sold as 2-manifolds with boundary do not fulfill our definition of a manifold. The problem lies in the sharp corners, the corner points do not allow the definition of a smooth border map. But as rounding up the corners any little bit solves this problem they can still be used for illustration purposes.

\subsubsection{Local Coordinates}

Any local map $\phi: U \subset \mathbb R^k \rightarrow M$ assigns \emph{local} coordinates to a manifold $M$. The values of the tuple $u=(u_1,...,u_k)\in U$ are called the local coordinates of the point $\phi(u)$, in the map $\phi$. A classic example for local coordinates is to parametrize the sphere by two angles using a map like
\[\phi(a,b) = (sin(a)sin(b), sin(a)cos(b), cos(a)).\]
This assigns coordinates $(a,b)$ to the sphere in the same way as longitude and latitude are used as coordinate for the world.\footnote{But note that you can not parametrize the whole sphere at once if the source domain of your map is an open set and the map has to be injective.}

Functions $f:M \rightarrow V$ that map the manifold to some arbitrary space $V$, can also be expressed in the local coordinates given by  a map $\phi$. This means that you consider $f \circ \phi : U \subset \mathbb R^k \rightarrow V$ instead of $f$, such that you can use the local coordinates $(u_1,...,u_k)$ as parameters of $f$ instead of the position on the manifold. %(Image : local coordinates..) 

	
\subsection{Tangential Spaces}		
\begin{figure}[tb]
\begin{center}
\includegraphics[width=8cm]{imgs/2_1_tangent.eps}
\end{center}
\caption{A curve $M$ parametrized by $\alpha$, together with the vector $\alpha'$ and the tangential space $T_{\alpha(t_0)}M$ at $t_0$.}
\label{fig::2_1_paramCurve}
\end{figure}


Manifolds have tangential spaces.
For curves calculating tangents is easy. Given a parametrization $\alpha(t): I \subset \mathbb R \to \mathbb R^n$ of some curve $M$, then $$\alpha'(t) = (\alpha_1'(t),\alpha_2'(t),...,\alpha_n'(t))$$ is a tangential vector of the curve at the position $\alpha(t)$. 		While the length of $\alpha'(t)$ at the point $\alpha(t)$ depends on the parametrization $\alpha$,  the \emph{tangential space} $$ T_{\alpha(t)}M = span(\alpha'(t)) = \{x \in \mathbb R^n: x = c \cdot\alpha'(t), c \in \mathbb R\}$$ does only depend on the position $\alpha(t)$ on the curve, as depicted in Figure \ref{fig::2_1_paramCurve}. 
On a $k$-dimensional manifold $M$, the tangential space $T_p M$ at a point $p$ can be characterized by any of the following ways:

\begin{enumerate}
	\item The tangential space is the vector space $T_pM\subset \mathbb R^n$ such that the affine linear space $p + T_p M$ approximates the surface in the best way, locally at $p$.
	\item The tangential space is the vector space $T_pM\subset \mathbb R^n$ made up by the tangents of all curves on the surface that go through $p$.
	\item For a given parametrization $\phi: \mathbb R^k \to \mathbb R^n$, $\phi(u) = (\phi_1(u),...,\phi_n(u))$ the tangential plane $T_{\phi(u)}M$ is given by
			\[span(\frac{\partial \phi} {\partial u_1},..., \frac{\partial \phi} {\partial u_k}) = span(\begin{pmatrix}
	\frac{\partial \phi_1} {\partial u_1} \\
	\frac{\partial \phi_2} {\partial u_1}\\
	\vdots\\
	\frac{\partial \phi_n} {\partial u_1}
\end{pmatrix},...,\begin{pmatrix}
	\frac{\partial \phi_1} {\partial u_k} \\
	\frac{\partial \phi_2} {\partial u_k}\\
	\vdots\\
	\frac{\partial \phi_n} {\partial u_k}
\end{pmatrix}),\]
as depicted in Figure \ref{fig::2_1_mapping_coords} for a 2-manifold. Here the role of the restriction that maps have a Jacobi matrix with full rank is clear, as it means that the partial derivatives are linearly independent.
\end{enumerate}

\begin{figure}[tb]
\begin{center}
\includegraphics[height=4.5cm]{imgs/2_1_mapping_coords.eps}
\end{center}
\caption{A map $\phi$ on a 2-manifold $M$ is used to determine the tangential space $T_{p}M$ at some point $p$.}
\label{fig::2_1_mapping_coords}
\end{figure}

The tangential space is defined for every point of a manifold. A manifold then is an object with vector spaces glued to all positions. While the tangential spaces $T_pM$  are well defined, the choice of a basis for $T_pM$ is open. Maps $\phi$ induce bases $\frac{\partial \phi}{\partial u_i}$, therefore a map can be used not only to parametrize the manifold but also the tangential spaces. But note that, just as it is sketched in Figure \ref{fig::2_1_mapping_coords}, these vectors are in general not orthogonal or normalized. 

\subsection{Orientations}

The orientation of volumes and manifolds is, together with the border operator, the geometric property that plays the most important part in EC and DEC. But what is orientation? Orientation is to assign signs to volumes. A volume can be positive or negative. To decide which it is, you need a reference -- you can only say how a volume is oriented relative to something.

For a vector space you can encode orientation in the ordering of basis vectors. Two ordered bases $v_1,...v_k$ and $w_1,...,w_k$ describe the same orientation if the matrix that describes the change of bases has a positive determinant. If a base is chosen, the determinant can also be used to measures the signed volume spanned by a set of vectors.

In the previous section we introduced tangential spaces and emphasized that every point gets its own proper tangential vector space. Tangential spaces of points that are close together are very similar and it makes sense to ask them to have the same orientation. A manifold is oriented by orienting its tangential spaces.

Local maps provide bases for tangential spaces. One single parametrization induces consistent orientations to the tangential spaces of all points it hits. Therefore we say that a manifold is oriented if all tangential spaces are oriented consistently:

\begin{definition}[Oriented Manifold] A manifold is \emph{orientable} if there exists a set of maps $\mathcal A = \{\phi: U_\phi \to \phi(U_\phi) \subset M\}$ such that the maps describe the whole manifold and any two maps $\phi$, $\psi$ which describe a common patch $\psi(U_\psi) \cap \phi(U_\phi)$ result in the same orientations, i.e., the base change matrix $C$ from the base formed by the columns of the Jacobian $D\phi$ to the base formed by the columns of $D\psi$ has a positive determinant,
\[det(C) >0.\]
A manifold is \emph{oriented} if for all tangential spaces a consistent orientation has been chosen.

\end{definition}

For 2-manifolds in $\mathbb R^3$, orienting a surface is equivalent to consistently choosing a surface normal, as tried in Figure \ref{fig::2_1_mobius}.

\begin{figure}[t]
\begin{center}
\includegraphics[width = 6cm]{imgs/2_1mobius.eps}
\caption{The Moebius strip, the pathological example of a non-orientable manifold.}
\label{fig::2_1_mobius}
\end{center}
\end{figure}

\subsection{The Border Operator}
The border operator describes a special geometric operation for manifolds. 
We denote the border of a manifold $M$ by $\delta M$ and call $\delta$ the border operator. From the way maps were defined at border points follows that the border of a manifold is again a manifold, and the dimension is decreased by one. Also, the border of a manifold always is a manifold without border, as can be seen with the interior of spheres or tori (see Figure \ref{fig::2_1_manifold}). This border of a border of a manifold is empty,
\[\delta\delta M = \emptyset.\]

A central point  is that an oriented manifold induces an orientation to its border, as sketched in Figure \ref{fig::2_1_borderManifold}. Therefore we can define the border operator $\delta$ such that it takes an oriented manifold and returns an oriented manifold with an induced orientation.  What follows is a short technical description of how the induced orientation on the border is defined formally. 

 As the orientation of a manifold is defined by the orientation of its tangential spaces we need to take a closer look at the tangential spaces of bordered manifolds.
At boundary points two tangential spaces are present. One is the tangential space of the manifold $T_pM$ and is $k$-dimensional, the other one is the $k-1$ dimensional tangential space of the border manifold $T_p \delta M$, as depicted in Figure \ref{fig::2_1_borderManifold} (left) . Inducing an orientation to the border means inducing an orientation to $T_p\delta M$ using the orientation of $T_p M$. This happens by defining normals on the border.

\begin{figure}
\begin{center}
\includegraphics[width = 13cm]{imgs/2_1_borderedManifold_combined.eps}
\end{center}
\caption{On bordered manifold two tangential spaces $T_pM$ and $T_p\delta M$  are present at boundary points. $N$ is the outward pointing border normal (left image). The base $(b_1,b_2)$ is oriented according to the manifold. The induced border orientation is given by a vector $\widetilde{b}$ such that  $(N, \widetilde{b})$ is oriented like $b_1,b_2$.}
\label{fig::2_1_borderManifold}
\end{figure}

For any border point one can define a border normal $N$. The border normal $N$ is the vector in $T_p M$ with:
\begin{itemize}
\item $N$ is orthogonal to $T_p \delta M$
\item $N$ has length 1
\item $N$ points outside
\end{itemize}
%Pointing outside is defined formally using the map $h$ at the border; $Dh$ is a linear bijective map from $\mathbb R^k$ to $T_pM$, so $N$ can be pulled back to $\mathbb R^k$ and it points 'outside' if the $k$th component of $Dh^{-1} N$ is negative \note{(Image?)}.

Orientation is encoded in the enumeration of basis vectors. If $b_1,...,b_k$ gives the orientation of $T_pM$,
a basis $\widetilde{b_1},...,\widetilde{ b}_{k-1}$ of the tangential space of the border $T_p\delta M$ is oriented according to the manifold, if prepending the normal $N$ to the basis leads to a basis $(N,\widetilde{b_1},...,\widetilde{ b}_{k-1})$ that has the same orientation as $b_1,...,b_k$. This is illustrated in Figure \ref{fig::2_1_borderManifold}.


\subsection{Differential Calculus on Manifolds}
\label{sec::2_derivativesOnMF}
So far we have seen tangential spaces, orientations and the border operator. The last property of smooth manifolds that needs to be covered is that manifolds allow differentiation to be done \emph{on} them. In this section we explain how mappings $f:M\to M'$ between two manifolds $M$ and $M'$ are differentiated \emph{on} the manifolds.  With this, manifolds become  more than geometric objects; they become spaces where differential calculus is possible, just as it is on $\mathbb R^n$. %The manifolds get a \textbf{differential structure}.

\subsubsection{Total Derivative}

\begin{figure}
\begin{center}
\includegraphics[height= 6cm]{imgs/3_1_manifoldDerivative.eps}
\end{center}
\caption{Construction of the total derivative of a real valued function $f$ defined on a manifold locally parametrized by $\phi$. $Df$ at a point $p$ is a linear mapping from the tangential space $T_p M$ to $\mathbb R$.}
\label{fig::3_1_manifoldDerivative}
\end{figure}

Given a manifold $M$ and a function $f: M \rightarrow \mathbb R^n$, what is the total derivative of $f$? We want the derivative to be something very similar to the total derivative $Dh$ of a function $h: \mathbb R^k \rightarrow \mathbb R^n$. In this case $Dh$ is the linear mapping that locally approximates $h$ and can be used to give the directional derivative for a direction $v$. For a fixed $p$, the Jacobian $Dh(p)$ is a $n\times k$ matrix; with $t \in \mathbb R$ close to 0 and $v \in \mathbb R^k$ we have 
\[h( p + t\,v) \approx h(p) + Dh(p) \cdot t\,v.\]
We want the same for functions $f$ on manifolds: $Df$ should be a linear mapping that maps a direction to the vector that describes the change of $f$ when going in that direction. But a direction on a manifold at some position is a tangential vector. Therefore, the differential $Df$ is a mapping from the \emph{tangential spaces} to vectors.

We can express the idea that $Df\cdot v$ describes the change of $f$ in the direction $v$ readily by using a curve $\alpha (t)$ with a tangent $\frac{\partial \alpha(0)}{\partial t} = v$ in the wished direction $v$, as depicted in Figure \ref{fig::3_1_manifoldDerivative} :
\begin{equation} Df \cdot v := \frac{\partial}{\partial t} f(\alpha(t)) \label{eq:2_1_derivativeDef}\end{equation}
As $f(\alpha(t))$ is simply a function $\mathbb R \rightarrow \mathbb R^n$ we know how to calculate the right hand side of Equation \ref{eq:2_1_derivativeDef}. We can also express the derivative in the local coordinates given by a parametrization $\phi(u_1,...,u_k)$.


\begin{figure}
\begin{center}
\includegraphics[width= 13cm]{imgs/3_1_manifoldDerivative2.eps}
\end{center}
\caption{Two 2-manifolds $M$ and $M'$ with local parametrizations $\phi$ and $\psi$ and a mapping $f: M \rightarrow M'$. The total derivative $Df$ at a point $p$ is a linear mapping from the tangential space $T_p M$ to $T_{f(p)} M'$. $D\phi$ and $D \psi$ parametrize the tangential spaces and $Df$ can be represented as a $2\times 2$ matrix.}
\label{fig::3_1_manifoldDerivative2}
\end{figure}


As we have seen, a parametrization provides a base for the tangential space, namely 
\[\frac{\partial\phi}{\partial u_1},..., \frac{\partial\phi}{\partial u_k}.\] 
We then express $\alpha$, $f$ and the tangential vector $\alpha'$ in the map $\phi$:
\begin{eqnarray*} \alpha(t) &=& \phi(u_1(t),...,u_k(t)),\\
\alpha'(t) &=& \frac{\partial\phi}{\partial u_1} u_1' + ... + \frac{\partial\phi}{\partial u_k} u_k',\\
 f(u_1,...,u_k) &=& f(\phi(u_1,...,u_k)) \\
 &=& f_1(\phi(u_1,...,u_k)),...,f_n(\phi(u_1,...,u_k)). \end{eqnarray*} 
Then Equation \ref{eq:2_1_derivativeDef} becomes
\[Df \cdot \alpha'(t) = \frac{\partial}{\partial t}f(u_1(t),...,u_k(t)) = (\frac{\partial f}{\partial u_1},..., \frac{\partial f}{\partial u_k}) \cdot \begin{pmatrix}
	u_1' \\ \vdots \\ u_k'\end{pmatrix}.\]
Therefore, $Df$ \emph{in the local coordinates given by $\phi$} is given by the $ n \times k$ matrix $(\frac{\partial f}{\partial u_1},..., \frac{\partial f}{\partial u_k})$ and if some tangential vector $v$ is described in the same map $v = v_1 \frac{\partial \phi}{u_1} +...+ v_k \frac{\partial \phi}{u_k}$:
\[Df \cdot v = (\frac{\partial f}{\partial u_1},..., \frac{\partial f}{\partial u_k}) \cdot \begin{pmatrix}
	v_1 \\ \vdots \\ v_k\end{pmatrix}.\]


\subsubsection*{Mappings between Manifolds}
\label{sec:derivativeBetweenMfs}
We can also consider the total derivative of mappings going from one manifold $M$ to an other manifold $M'$,
\[f:M\to M',\]
as shown in Figure \ref{fig::3_1_manifoldDerivative2}. We look again at Equation \ref{eq:2_1_derivativeDef}:
\[Df \cdot v := \frac{\partial}{\partial t} f(\alpha(t)).\]
Note that $f(\alpha(t))$ is a curve on $M'$ and $\frac{\partial}{\partial t}f(\alpha(t))$ therefore is a tangential vector to this curve and lies in $T_{f(\alpha(t))}M'$. Then $Df\cdot v$ has to be a vector in the tangential space of $M'$. This means that the derivative $D_pf$ at some point $p$ is a linear mapping from the tangential space $T_pM$  to the tangential space $T_{f(p)} M'$, i.e., 
\[D_p f = T_p M \rightarrow T_{f(p)} M'.\] 

If $M$ is a $k$-manifold and $M'$ a $l$-manifold, $Df$ can be expressed as a $l\times k$ matrix, described relatively to two sets of local coordinates $\phi \rightarrow M$ and $\psi \rightarrow M'$.

\subsection{Summary}
Manifolds are objects with geometric features, for example the orientation of volume and the definition of a border operator, but they also allow differential calculus on their surfaces. Tangential spaces play a crucial role by allowing local properties to be pointwisely defined on them and differentiation leads to linear mappings between tangential spaces.


\newpage
\section{Discrete Manifolds}
\label{sec::2_discreteManifolds}
In the last sections we had a look at the geometric objects exterior calculus will be defined on, i.e. smooth surfaces and manifolds. The next step is to introduce the discrete analogues we want to do computations with: triangle meshes, or more generally simplices and simplicial complexes. Simplices are for example points (0-dimensional), lines (1-dimensional) triangles (2-dimensional) and tetrahedra (3-dimensional). Simplicial complexes are `meshes' made out of them. The definitions are taken from \cite{DMK08} and \cite{FRANKEL11}.

\subsection{Simplices and Simplicial Complexes}

\begin{figure}[t]
\begin{center}
\includegraphics[height= 2cm]{Imgs/2_3_simplices.eps}
\end{center}
\caption{A 0-simplex (point), 1-simplex (line), 2-simplex (triangle), and 3-simplex (tetrahedron). }
\label{fig::2_3_simplices}
\end{figure}

A $k$-simplex is the most basic geometric object with a $k$-dimensional volume: the convex hull of $k+1$ points, as depicted in Fig. \ref{fig::2_3_simplices}. No point should lie in the convex hull of the others; else no $k$-dimensional volume is spanned and the simplex is called degenerated.

\begin{definition}[Simplex] A non degenerated $k$-simplex is the convex hull of $k + 1$ points $p_1,...,p_{k+1}$, where the vectors $p_2 -p_1, p_3,-p_1, ..., p_{k+1} -p_1$ are linearly independent. It is represented as a tuple of its corner vertices $\{p_1,...,p_{k+1}\}$.
\end{definition}

Every simplex has faces of various dimensions: any combination of $l+1$ of its corner vertices forms an $l$-dimensional face. For example a tetrahedron has 4 2-dimensional faces (triangles), 6 1-dimensional faces (edges) and 4 0-dimensional faces (vertices),see Figure \ref{fig::2_3_simplices}. A 4-simplex would have 5 tetrahedral faces and so on.

Out of simplices one can build simplicial complexes, in the same way as meshes are built out of triangles. The restrictions are the usual: the interior of any two simplices should not overlap, and if the intersection of two simplices is not empty, the intersection has to be a face of both simplices. A simplicial complex then is a list of simplices, following these restrictions. 

If a simplicial complex contains a  $k$-simplex $\sigma$, we also demand that all faces of $\sigma$ are part of the simplicial complex. This is not just a tedious technical detail;  we explicitly want to associate different values to all faces of simplices. In a triangle mesh, for example, we will need to keep track not only of triangles and vertices but also of the edges.

\begin{definition}[Simplicial Complex]
A simplicial  complex is a collection $\kappa$ of simplices, such that if a simplex is contained in $\kappa$, all its faces are too. Furthermore the intersection of any two simplices in  $\kappa$ is either empty or a common face.
\end{definition}

Lastly we do not want our discrete manifolds  to have the analogue of dangling triangles (Figures \ref{fig::2_2_dangling} and \ref{fig::2_2_dangling2}). To ensure this formally one has to make a restriction that is similar to the definition of manifolds. Just as we ensured that a $k$-manifold locally looks like $\mathbb R^k$ or $\mathbb H^k$, we want to make sure our discrete manifold looks locally like either a $k$-dimensional ball or a $k$-dimensional half-ball, as depicted in Figure \ref{fig::2_2_homeoToBall}. This gets rid of dangling things.

\begin{figure}
\begin{center}
\includegraphics[height=2.5cm]{imgs/2_2_dangling.eps}
\caption{These are not discrete 2-manifolds: the first mesh has a dangling triangle, the second mesh has a `wheel' and is not locally equivalent to a plane, the same holds for the third mesh.}
\label{fig::2_2_dangling}
\end{center}
\end{figure}

\begin{definition}[Discrete Manifold]
A $k$-dimensional discrete Manifold is a simplicial complex where for every vertex in $\kappa$ the union of all incident simplices is equivalent to a $k$-dimensional ball or a $k$-dimensional half ball.
\end{definition} 

On discrete manifolds we can define orientations and a border operator with the same geometric meaning as on smooth manifolds.

\subsection{Orientations}
\label{subsec:SC_orientations}
Orientations can be treated on discrete manifolds similarly as on continuous manifolds. Orientations are quite of  practical importance and a notorious source of switched sign errors in the context of discrete exterior calculus. 

We can assign one of two orientations to a simplex of any dimension, meaning that the volume represented by the simplex should be considered as positive or negative. While we coded orientation before in the enumeration of basis vectors, we encode the orientation of simplices in the enumeration of their corner vertices.  For edges it is the most intuitive what this means: we assign a direction to the edge $\{p_1,p_2\}$ by saying that the first vertex listed is the start vertex of the edge. Note that for an edge or any geometric object there is not a strict `positive' or a `negative' orientation; we can only say how something is oriented relative to something else. For example the edge $\{p_1,p_2\}$ is oriented negatively to the edge $\{p_2,p_1\}$; this is noted as
\[-\{p_1,p_2\} = \{p_2,p_1\}.\]
So the orientation of a $k$-simplex depends on the way its corner vertices are enumerated. Two enumerations of corner vertices result in the same orientation if they are related by an even permutation. A permutation is called even, if it can be reproduced by switching pairs of vertices an even number of times. E.g.
\[\{a,b,c,d\} = \{c,a,b,d\}\]
\[\{a,b,c,d\} \rightarrow \{c,b,a,d\}\rightarrow \{c,a,b,d\},\]
where we get from the first tuple to the second by swapping  $a$ and $c$ and then $a$ and $b$, i.e. with two swaps. Alternatively, determinants can be used to determine the sign of a permutation; just calculate the determinant of the permutation matrix
\[\{a,b,c,d\} \rightarrow \{c,a,b,d\},\]
\[\begin{pmatrix}c\\a\\b\\d \end{pmatrix}=\begin{pmatrix} 0 & 0 & 1 &0 \\ 1 &0&0&0 \\ 0&1&0&0 \\ 0&0&0&1 \end{pmatrix}\begin{pmatrix} a\\b\\c\\d \end{pmatrix}.\]
Or again you can use the simplex to induce a base to the affine vector space it is aligned to
\[p_1 -p_2,...,p_{k}-p_{k+1},\]
and two vertex enumerations induce the same orientation if the induced bases have the same orientation. This also shows that defining the orientation of a simplex by looking at the ordering of its corner vertices amounts to the same as orienting volumes by choosing bases.

\begin{figure}%
\begin{center}
\includegraphics[height = 4cm]{imgs/2_2_homeoToBall.eps}%	
\end{center}
\vspace{ -0.5cm}
\caption{Discrete manifolds are asked to be locally `equivalent' to balls or half-balls. This is meant in the sense of homeomorphism: there should exist a homeomorphism, a continuous bijective mapping, between 1-neighborhoods and either balls or half-balls. In (a) the 1-neighborhood of the marked vertex is homeomorph to a 2D ball, i.e. a disk, in (b) the 1-neighborhood is homeomorph to a half-ball, and in (c) there exists no homeomorphism to either a half-disk or a disk.}%
\label{fig::2_2_homeoToBall}%
\end{figure}

\begin{figure}
\begin{center}
\includegraphics[height=2.5cm]{imgs/2_3_danglingTetrahedra2.eps}
\caption{If tetrahedra are not connected by two dimensional faces, they are `dangling'.}
\label{fig::2_2_dangling2}
\end{center}
\end{figure}


One exception are vertices or 0-simplices $\{v_0\}$, where orientation is not encodable in the enumeration of the vertex. We need to assign orientations to single points too and say that $-\{v_0\}$ is the negatively oriented version of $\{v_0\}$. Orientation is `imprinted' on the point. The best way of thinking of orientation is that orientation adds a sign to volumes. A negatively oriented point is then a point whose $0$-dimensional volume is negative. The $0$-dimensional volume of any single point is defined to be either $1$ or $-1$ and the 0 dimensional volume of a point set is $\#positive\; points - \#negative\; points$.

As long as you stick with calculations in $\mathbb R^3$ it stays pretty simple to determine if two orientations of a simplex are the same, if you stick with triangle meshes it is trivial. Just make sure you always remember to respect orientations. In Section \ref{sec::2_handsOnSimplicialComplexes} we will also come back to the question of how to compute relative orientations in practice.


\subsection{Formal Sums, k-Chains and Vectors}
\label{subsec::formalsums}

\begin{figure}
\begin{center}
\includegraphics[width = 12cm]{imgs/2_2_formalsum.eps}
\end{center}
\caption{Two sets of edges expressed as formal sums that get summed up.}
\label{fig::2_2_formalsum}
\end{figure}

Single oriented $k$-simplices are represented as ordered tuples of vertices, optionally with a sign to denote switched orientation. A set of $k$-simplices can then be described by a \emph{formal sum}, and is also called a \emph{simplicial $k$-chain}. For example the set of the three edges $\{v_0, v_1\}$, $\{v_1, v_2\}$ and $\{v_2, v_0\}$ written as a formal sum is
\[\{v_0,v_1\} + \{v_1,v_2\} +\{v_2,v_0\}.\]
Additionally, a simplicial $k$-chain is allowed to have multiple `copies' or negatively oriented copies of some simplex. This is taken account of by associating integer values from $\mathbb Z$ to the simplices, for example
\[3\{v_0,v_1\} -2\{v_1,v_2\} +\{v_2,v_0\}\]
would have three copies of $\{v_0,v_1\}$ and so on. The negative integers denote a change of orientation as described in the last section; the formal sum could be rewritten without the use of negative integers as
\[3\{v_0,v_1\} +2\{v_2,v_1\} +\{v_2,v_0\}.\]
Two $k$-Chains can be combined by summing them up,
$$c = \sum_{i=1}^N c_i \sigma_i,\;\;\; d= \sum_{i=1}^N d_i \sigma_i,$$
$$c+d = \sum_{i=1}^N (c_i + d_i)\sigma_i,$$
where the $c_i$ and $d_i$ are integers from $\mathbb Z$ and the $\sigma_i$ are $k$-simplices. This reflects that if in the set $c$ there are $c_i$ copies of $\sigma_i$ and in $d$ there are $d_i$ copies, then in the combined set $c+d$ there have to be $c_i+d_i$ copies of $\sigma_i$. In particular, oppositely oriented simplices cancel out, see also Figure \ref{fig::2_2_formalsum}.

Formal sums are one way to describe sets of oriented $k$-simplices. Equivalently the $k$-chains can be written as a vector of integers if a global enumeration of all $k$-simplices has been chosen and one reference orientation has been assigned to every simplex. For example consider the enumeration of vertices, edges and faces in Figure \ref{fig::2_2_complexenum}. Note that every kind of simplices is enumerated separately. With this fixed enumeration a set of vertices can be described as vector in $\mathbb Z^4$, a set of edges as a vector in $\mathbb Z^5$ and a set of faces as a vector in $\mathbb Z^2$. Generally a set of $k$-simplices can then be identified with an integer vector with the dimension of the number of occuring $k$-simplices, i.e. a vector from $\mathbb Z^{\#k\text{-}simplices}$. The $i$th value in such a vector then denotes the number of occurences of the $i$th simplex. For example, using the enumeration of Figure \ref{fig::2_2_complexenum}, the formal sum $e_0 -e_1 + 2e_2$ would be described by the vector $(1,-1,2,0,0)$.

For computations the vector notation for $k$-chains is the most useful; but in order to use it a global enumeration and fixed reference orientations of the simplices need to be chosen first.

\begin{figure}%

\begin{center}
\def\svgwidth{13cm}
\input{imgs/2_2_complexEnumeration.eps_tex}
\end{center}
\vspace{-0.5cm}

\caption{A simple simplicial complex with a fixed enumeration of the occuring simplices; for each simplex a reference orientation has been chosen.}%
\label{fig::2_2_complexenum}%
\end{figure}

\subsection{The Border Operator}
\label{sec::2_borderOrientation}
Discrete manifolds or any set of simplices allow the definition of a border operator similar to the border operator defined on continuous manifolds. And just on continuous manifolds, an oriented discrete manifold induces an orientation to its border.

%We first introduce some notation: we can respresent collections of simplices as \emph{formal sums}, as depicted in Figure \ref{fig::2_2_formalsum}. The simplices are represented as tuples, a negaitve sign means a change of orientation. Simplices that are different from each other are not summed up, only if two tuples describe the same simplex but for orientation, the sum is taken. Particularly simplices of opposed orientation cancel out.

The border of a single $k$-simplex is  the following formal sum
\[\delta\{v_0,v_1,...,v_k\} = \sum_{j=0}^k (-1)^j\{v_0,...,\widehat{v_j},...,v_k\},\]
where  the $\widehat{v_j}$ means omitting $v_j$. This expresses that the border of the simplex is a set of $k-1$ simplices. The orientations they get are the ones that the simplex induces. Note that prepending the omitted vertex $v_j$ to $(-1)^j\{v_0,...,\widehat{v_j},...,v_k\}$ leads to a simplex with the orientation of $\{v_0,v_1,...,v_k\}$. This is consistent with the way we defined that orientation should be induced to borders of smooth manifolds. 

For example the border of a triangle $\{a,b,c\}$ is \[\{b,c\} -\{a,c\} + \{a,b\} = \{a,b\} + \{b,c\} + \{c,a\},\] just as it should be.

But if we can take the border of single simplices, we can also take the border of a set of simplices or of discrete manifolds; it is simply the formal sum of the borders of the $k$-simplices the discrete $k$-manifold is made out of. Formally, the border operator maps simplicial $k$-chains to simplicial $k-1$-chains and is a linear mapping; if $c$ and $\widetilde{c}$ are two $k$-chains, the border operator fulfills
\[\delta (c+\widetilde{c}) = \delta c + \delta \widetilde{c}.\]
Another point is that, as you see in the Figure \ref{fig::2_2_borderUnoriented}, the border operator gives a `wrong' result when a discrete manifold is oriented inconsistently. 

\begin{figure}
\begin{center}
\includegraphics[width = 6cm]{imgs/2_2_borderUnoriented.eps}
\end{center}
\caption{The border operator that respects orientation only makes sense with oriented discrete manifolds. Orientation of faces is depicted by an arrow that says what orientation a simplex induces to its border.}
\label{fig::2_2_borderUnoriented}
\end{figure}

\subsubsection{The Border Operator as a Matrix}
Instead of representing the border operator as a formal sum we can write it as a matrix, provided that an enumeration and reference orientation of the simplices has been chosen. The sets of $k$-simplices are described by integer verctors as introduced in Section \ref{subsec::formalsums}. The border operator $\delta_k$ \emph{for $k$-simplices} is a linear mapping from the simplicial $k$-chains to the simplicial $k-1$-chains and therefore is a linear mapping  \[\delta_k:\mathbb Z^{\#k\text{-}simplices} \rightarrow \mathbb Z^{\#k-1\text{-}simplices}.\]
Note that we explicitly make a distinction between the border operators for simplices of different dimensions and annotate the border operators with a subscript $k$ that describes the dimensionality of the simplices the border operator can be applied to. 

These border operator matrices form the backbone of DEC and they will be used excessively throughout this thesis. Also it is very important to understand that they describe a purely geometric operation, namely the operation of going from a $k$-dimensional set of simplices to a $k-1$-dimensional set describing the boundary. The border operator matrices are relatively easy to compute and store the relative orientations of $k$-simplices and $k-1$-simplices. Consider the following toy-example: we look at the simplicial complex with fixed simplex enumerations and reference orientations from Figure \ref{fig::2_2_complexenum}, which is inlined here once more for convenience.

%If all simplices occurring in a complex are enumerated and have a fixed orientation, the border operator can be expressed as a matrix, the incidence matrix. A set of $j$ simplices is represented as vector of integers of dimension $\#(j\;simplices)$. The $k$-th entry of this vector represents the number of times the $k$-th $j$-simplex occurs. 
%For example in the following enumeration
\begin{center}
\def\svgwidth{13cm}
\input{imgs/2_2_complexEnumeration.eps_tex}
\end{center}
%the set of edges $e0 - e1 + e2 $ is represented by the vector $(1,-1,1,0,0)$.
Here we have two 2-simplices (triangles), five 1-simplices (edges) and four 0-simplices (vertices). There are two border operator matrices: one to compute boundaries of $2$-dimensional sets and one to compute the boundary of 1-dimensional sets. The first one, $\delta_2$, maps 2-chains to 1-chains and therefore has to be a linear mapping from $\mathbb Z^2$ to $\mathbb Z^5$, i.e. a $5 \times 2$ matrix.

The first row of $\delta_2$ describes the result of applying the border operator to the first face $f_0$, i.e. $\delta_2 \cdot (1,0)^T$. The boundary of the face $f_0$ is the formal sum $e_0 + e_2 - e_1$, which is described by the integer vector $(1,-1,1,0,0)^T$. The second row $\delta_2 \cdot (0,1)^T$ of $\delta_2$ describes the boundary of $f_1$ and is given by the formal sum $-e_2 -e_4 + e_3$, i.e. $(0,0,-1,1,-1)^T$. Therefore the complete matrix is given by
\[\delta_2 = \begin{pmatrix}
1 & -1 & 1 &0&0\\
0& 0& -1 & 1 & -1
\end{pmatrix}^T.\]
The other border operator matrix, $\delta_1$, is a linear mapping from simplicial $1$-chains to $0$-chains and therefore is a $4 \times 5$ matrix. Here the $i$th matrix row describes the border of the $i$th edge $e_i$; for example the border of the edge $e_0$ is the formal sum $v_2 -v_0$ and therefore the first row of $\delta_1$ has to be $(-1,0,1,0)^T$. The complete matrix is given by
\[\delta_1 = \begin{pmatrix}
-1&-1&0 &0 & 0\\
0&1&1 &1 & 0\\
1&0&-1 &0&1\\
0&0&0&-1&-1\\
\end{pmatrix}. \]

Of course, the border matrices can be applied to more complex chains. For example the border of the line segment $e_0 -e_4 + e_3$ or would be given by
\[\begin{pmatrix}
-1\\ 1\\ 0 \\0 
\end{pmatrix} = \delta_1\begin{pmatrix}
1\\0\\ 0\\1\\-1
\end{pmatrix},\]
which is $-v_0 + v_1$, saying that $v_0$ is the `start' and $v_1$ the `end' border of the line. 

Generally, for a $k$-complex there is a total of $k$ border matrices: the border operator $\delta_1$ for sets of 1-simplices (edges), the operator $\delta_2$ for 2-simplices (triangle faces), $\delta_3$ for 3-simplices and so on. %We will always make the difference between these border operators; we add a $j$ as subscript to the border operator of $j$-simplices : $\delta_j$. 
The entry $(i,j)$ in a border matrix is the relative orientation of the two simplices concerned. In the example above $\delta_1(0,1) = -1$ because the vertex $v_0$ is oriented negatively relative to the edge $e_1$, considering the border induced orientation.

\subsection{Oriented Discrete Manifold}
\label{sec::2_orientedDiscreteMF}
Lastly we can not only orient single simplices, but also a whole discrete manifold. This leads to oriented discrete manifolds, which are the discrete analogue to smooth oriented manifolds. 

The orientation of a volume is strongly linked to the orientation of borders. For convenience we will define well orientedness of a discrete manifold using the border orientations.
Two $k$-simplices that share a $k-1$ dimensional face are oriented consistently exactly if the induced orientation of this face is opposed for both $k$-simplices, as depicted in Figure \ref{fig::2_2_borderUnoriented}.
A $k$-manifold is oriented if all-$k$ simplices are oriented consistently. 

\subsection{Summary}

\begin{figure}%
\begin{center}
\includegraphics[height=3cm]{imgs/2_2_snoothVSdiscreteBorderOp.eps}	
\end{center}
\caption{Applying the smooth border operator to a $k$-manifold returns a $k-1$-manifold, the same holds in the discrete setting. Also for \emph{oriented} smooth and discrete manifolds $\delta\delta = 0$ holds.}%
\label{fig:2_2_snoothVSdiscreteBorderOp}%
\end{figure}
Discrete manifolds are geometric objects made out of simplices and allow the definition of orientation and border operators, just like smooth manifolds. Also, applying the discrete border operator twice to an oriented discrete manifold leads to an empty set:
\[\delta_{j-1}\delta_{j} = 0.\]
This mirrors the property of the smooth border operator. The smooth and the discrete border operator are depicted schematically in Figure \ref{fig:2_2_snoothVSdiscreteBorderOp}.

Subsets of $j$-simplices of a discrete $k$-manifold can simply be represented by vectors and the border operators by matrices.  Up to now only the geometry of smooth manifolds has been discretized; to develop an analogue to differential calculus on manifolds we will first need to introduce differential forms.


\newpage
\section{Implementation Notes}
This chapters provides a guideline of what needs to be implemented to get DEC and the later applications up and running. The components needed are described and some of the more tricky details are mentioned.

\subsection{A Word on Sparse Matrices}
The point of DEC is to reformulate differential equations using sparse matrices. Therefore any implementation of DEC is somewhat centered around sparse matrices.

If you plan to implement your DEC framework you should start by looking for a sparse matrix solver. For all results in this thesis the sparse solver from the Pardiso-Project of the University of Basel has been used as a black box solver \cite{pardiso1}. Unfortunately it is not freeware but any other sparse solver will do as well.

The Pardiso solver uses the so-called Yale format. The Yale format uses 3 vectors to describe an arbitrary $n\times m$ matrix $A$. The first vector $a$ stores all non-zero values of $A$, enumerated by row. The second vector $ja$ stores the column indices of the non-zero values, again enumerated by row. The third vector $ia$ stores for every row the index $ind$, such that $a(ind)$ and $ja(ind)$ are the value and the column of the first element in the row. Additionally one appends the length of the vector $a$ to $ia$.

For example 
\[\begin{pmatrix}
1 & 0 & 0 &3 \\
 0 & 0 & 0 &2 \\
 0 & 4&2&0
\end{pmatrix} \Rightarrow \begin{cases} a &= [1,3,2,4,2]\\ ja &= [0,3,3,1,2] \\ ia &= [0,2,3,5]  \end{cases}\]
Iterating over the values and indices of the $k$th row then amounts to
\begin{algorithmic}
\FOR{i = ia(k):ia(k+1)}
	\STATE out $\gets (k,ja(i))$   //the index pair
	\STATE out $\gets a(i)$  //of this value
\ENDFOR
\end{algorithmic}

\subsection{Implementing Simplicial 2-Complexes}
\label{sec::2_handsOnSimplicialComplexes}
The applications in this thesis focus on 2-complexes, i.e., classical triangle meshes. You might not need any more general implementation. General $k$-complexes are treated separately in the next section. 

For DEC we need the complete geometric information of meshes; we have to explicitly keep lists of vertices, edges and faces, the full information about their incidence and border relations, as well as their assigned orientations. 

For 2D meshes a winged edge structure is a convenient choice of representation. In a winged edge structure you have the following three objects: vertices, edges and faces, as described in Figure \ref{figs::2_1_wingedEdge}. Edges are stored once, with an arbitrary chosen orientation. With this information present it is easy to do things like iterating over the incident edges or faces of a vertex or iterate over the edges of a border component following the orientation of the mesh.

Note that, even though the incidence information is stored in a winged mesh structure, for DEC applications the discrete border operator matrices $\delta_1$ and $\delta_2$ still need to be computed.

\begin{figure}[tb]
\begin{center}
\includegraphics[height = 3cm]{imgs/2_1_wingedEdge.eps}%	
\vspace{0.5cm}

\includegraphics[height = 4cm]{imgs/2_2_wingedEdge2.eps}
\end{center}
\caption{The information stored on a winged edge structure.}%
\label{figs::2_1_wingedEdge}%
\end{figure}


\subsection{Implementing Simplicial k-Complexes }

\begin{figure}[t]
	\begin{center}
	\includegraphics[height=4cm]{imgs/2_1_Complex.eps}%	
	\end{center}
	\caption{Implementation of a $k$-complex that uses tuple of ordered indices to characterize a simplex.}%
	\label{fig::2_1_Complex}%
\end{figure}

Chances are you do not need simplicial complexes of higher dimensions other than tetrahedral meshes embedded in $\mathbb R^3$. But one straight forward and for DEC suitable way to implement arbitrary $k$-complexes is to store lists of simplices and represent the incidence information explicitly with the sparse border matrices. The border operator matrices play a central role in DEC and need to be set up anyway.

Figure \ref{fig::2_1_Complex} depicts a possible implementation of $k$-complexes.  A $k$-complex consists of $k+1$ simplex lists and of the border matrices $\delta_1,...,\delta_k$. The vertices (0-Simplices) store their positions. A $j$-simplex is represented by a $j$-tuple of vertex indices, these are the indices of the vertices in the list kept in the $k$-complex data structure. The index tuples describing the simplices are sorted, i.e.,
\[(i_1,i_2,...,i_j): i_1 < i_2 <...<i_j.\] 
Sorted tuples facilitate the computation of the relative orientation of a $j-1$ simplex $(v_0,...,\widehat{v_l},..., v_{j})$ lying on the border of a $j$ simplex $(v_0,...,\widehat{v_l},..., v_{j})$. By the definition of the border operator from Section \ref{sec::2_borderOrientation} their relative orientation simply is:
\[relativeOrientation = (-1)^l\]
The orientation of most simplices of an oriented simplicial $k$-complex can be chosen arbitrarily and we choose the orientation that is defined by the sorted index tuples. Only the $k$-simplices of an oriented $k$-complex have to be oriented consistently. Therefore, they get an additional `orientation' variable in which is stored if the orientation induced by the ordered tuple is positive or negative relative to the orientation of the $k$-complex.


\subsubsection{Setting Up a k-Complex}
Setting up the border operator matrices $\delta_j$ for complexes of arbitrary dimensions can be a fuzz, as you need to excessively compute relative orientations of simplices. Supposing that you have a wireframe model of the $k$-complex, i.e., a list of vertex positions and a list of $k$-simplices, the full $k$-complex  with all incidence matrices can be set up in the following way:
\begin{enumerate}
	\item Reorder the index tuples of the $k$-simplices and adapt the orientation accordingly
	\item Generate all the $j$-simplex lists
	\item Set up the matrices $\delta_j$
\end{enumerate}


%But while it is easier to compute relative orientations if the indices of your simplices are sorted, you loose the ability to store arbitrary simplex orientations using the ordering of vertices. For all but the $k$ simplices this does not matter, even for oriented discrete $k$-manifolds, as there is nor 'right' or 'wrong' orientation and all that matters is that you consistently use the same orientation all the time. But for the $k$ simplices in an oriented $k$ manifold you really need to be able to chose the orientation, so you have to keep track of the orientation independently in an additional variable (as is done in Figure \ref{fig::2_1_Complex}).

\subsubsection{Reordering tuples: } When reordering the tuples in the first step, you need to keep track of how the orientation changes. This can be done using a so called \emph{inversion table}. Lets say the tuple $(i_1,...,i_n)$ is a permutation of the tuple $(1,2...,n)$. An inversion is an index pair $(i_l,i_k)$, where $l<k$ but $i_l >i_k$, i.e., the order of $i_l,i_k$ is inverted. The relative orientation of the simplex $\{i_1,...,i_n\}$ and the ordered simplex $\{1,2...,n\}$ is $(-1)^{\#inversions}$ where $\#inversions$ is the total number of inversions.

In the inversion table you count for every index in the tuple the number of elements on its left that are greater. Example: 
\begin{align*}
permuted \;\; tuple & &3,2,5,4,1 \\
inversions & & 0,1,0,1,4
\end{align*}
The first line represents the permuted indices the lower the number of inversions of every index. The total number of inversion is  1+1+4 =6 and the relative orientation of
$\{3,2,5,4,1\}$ to $\{1,2,3,4,5\}$ is $(-1)^6 = 1$.

\subsubsection{Setting up incidence matrices:}
After the steps 1 and 2, all simplices in the simplicial complex have a fixed orientation and a fixed position $ind$ in the simplex lists. Setting up the border matrix $\delta_j$ is then described in the listing in Figure \ref{alg:deltaj} 

\begin{figure}
%\begin{tabular}[c]

\begin{center}
\fbox{\parbox{10cm}{
\begin{algorithmic}
\FOR{ind1 = (0:$\# j$-$simplices$)}
	\STATE \emph{//get the \emph{ind1}th j-simplex:}
	\STATE $\{i_0,...,i_j\} \gets j$-$simplex[$ind1$]$   
	\STATE \emph{//Iterate of the bordering j-1 simplices}
	\FOR{l=(0:j)}
		\STATE \emph{//Find the index of the \emph{l}th bordering j-1 simplex}
		\STATE ind2 $\gets index(\{i_0,...,\widehat{i_l},...,i_j\})$ 
		\STATE \emph{//Store the relative orientation of the border simplex //and the simplex in $\delta_j$}
		\STATE $\delta_j($ind1,ind2$) \gets (-1)^l$
	\ENDFOR
\ENDFOR
\end{algorithmic}
}}
\end{center}
%\end{tabular}
\caption{An algorithm to set up the border operator matrix $\delta_j$.}
\label{alg:deltaj}
\end{figure}
		
\subsection{Simple Example Applications}
Using the border matrices you can easily check if a mesh is well oriented or compute the border components of a mesh.

	
\subsubsection{Orientation Test}
That a $k$-complex is oriented can be checked by looking at $\delta_k$. The condition we gave in Section \ref{sec::2_orientedDiscreteMF} for a $k$-complex to be oriented was the following: any $k-1$ simplex is either a border simplex, therefore being part of exactly one $k$-simplex, or it is part of two $k$-simplices, having once positive and once negative orientation. 

This is exactly the case if any column of $\delta_k$ has either exactly one entry, which is the case for boundary simplices, or two entries: a one and a minus one. 

\subsubsection{Check if a 2-Complex is a Discrete Manifold}
\begin{figure}[tb]
	\begin{center}
	\includegraphics[width=12cm]{imgs/2_3_danglingTeapot.eps}
	\end{center}
	\caption{A teapot mesh that on the first look seems to be a discrete borderless 2-manifold but turns out to be a mesh with border and dangling triangles, which makes it a non-manifold mesh and therefore not suited for some DEC applications. The deformed teapot on the right was generated using the area minimizing flow mentioned in the introduction, Section \ref{sec:1_example}}.
	\label{fig:2_badteapot}
\end{figure}

DEC applications might fail if a complex is not a discrete manifold. Two dimensional meshes used for displaying purposes are often not discrete manifold. To avoid singular matrices and to eliminate the possibility that bugs occur due to the ill-formedness of a mesh it is good to test a mesh for manifoldness before using it. The mesh in Figure \ref{fig:2_badteapot} is an example of a non-manifold mesh that leads to unexpected results.

In a winged edge mesh, finding dangling faces is fairly simple; at every vertex iterate over all edges and make sure that either exactly 2 or no edges have only one neighbor face.
	
\subsubsection{Finding Borders}
Given an oriented discrete $k$-manifold we want to find the $(k-1)$-complex that represents its border. This can be easily done by applying the border Operator $\delta_k$ to the $(1,1,1,1,1...)$ vector, which represents the whole manifold. The resulting vector then exactly represents the formal sum of $k-1$ simplices that describes the border manifold. If multiple borders are present the connected components of this border have to be computed in a separate step.
