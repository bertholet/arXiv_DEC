\chapter{(?) Differential Structure and Differential Calculus on Manifolds}
\note{Skip chapter}

\note{Rework this stupid chapter i never wanted to write- integrate it in the right chapters... because i dont want it to take any space at all because well, for one it just distracts from the actual goal of differential forms, but then differential forms can only be understood as a generalization of something if the something is known a priori which makes this stupid chapter kind of needed but i do not want it!!!!!!}
In the last chapter we already encountered a Laplacian defined on a 2D surface and considered derivatives of functions defined on a surface, especially when calculating the derivative of the surface normal field. We saw how the derivative $Df$ of a function is mapping between two tangential spaces. Differential Forms and External Calculus, which are the real topics of this thesis, are a generalization of classical calculus (gradients, curls etc) on manifolds.  So before we move on to the core topic of this Thesis we want to elaborate derivatives and other differential operators on manifolds just a bit more, to give you a better feeling of what we are generalizing later, such that you understand where the operators exactly 'live' and why we have calculus on manifolds.

This chapter is rather short and of a completely theoretical nature.

\section{Differentiability Revisited}
What we have seen up to here actually amounts to defining manifolds and providing a differential structure on them. With a differential structure manifolds become more than geometric objects that are measured and described. They become spaces on which you can define functions for which derivatives and other operations can be calculated directly ON the manifold. While all operations are defined by mapping everything back to $\mathbb R^n$, this differential structure exists on its own right.

The differential structure is 'glued' on a manifold using maps, as seen in the subsection Derivatives on Surfaces. Differentiability is then defined in the following way:

\begin{definition}[Differentiability] If $M^k$ and $N^l$ are two manifolds and $f: M^k \rightarrow N^l \subset \mathbb R^n$ is a continuous mapping we call $f$ differentiable if for every map $h:\mathbb R^k \rightarrow M^k$ the map $f \circ h :  \mathbb R^k \rightarrow \mathbb R^n$ is differentiable.
\end{definition}

As seen the differential of a function at some point is a linear mapping between two tangential spaces.
\[D_pf: T_p M^k \rightarrow T_{f(p)} N^l\]
A slightly more abstract way of viewing this is looking at $T M^k$ : the space of all tangential spaces, called the tangential bundle. Given a map $h:\mathbb R^k \rightarrow M^k$ we can get a local map of the tangential bundle 
\[h_* : \mathbb R^k \times \mathbb R^k \rightarrow T M^k\]
\[h_*(x,v) = (h(x), Dh\, v)\]
suggesting that the tangential bundle actually is $2k$-manifold. The differential of a function $f: M^k \rightarrow N^l$ then is a mapping between the two tangential bundles $T M^k$ and $T N^l$ which are a $2k$ and a $2l$ manifold.

For our first steps of differential calculus on manifolds we also need vector fields (which we will later generalize to Forms).  A vectorfield $\mathcal V$ on a manifold $M^k$ is simply the assignment of a vector from the tangential space to every point of $M^k$.
\begin{definition}[Vector Field]
A vector field is an assignment $\mathcal V: M^k \rightarrow T M^k$ with $\mathcal V(x) \in T_x M^k$.
\end{definition}
Using that the tangential bundle actually is a manifold with a differential structure we can ask from a vector field that it is smooth or differentiable.

We can then define a vectorfield in a coordinate/ map dependent way. We do this with a simple example taken from [Thomas Friedrich, Global Analysis].

\subsubsection{An example Vector Field} Taking $\mathbb R^2$ as a manifold parametrized by the identity map $\phi(u,v) = ID$, i.e. para-metrized with euclidean coordinates. The tangential spaces get the bases $\frac{\partial \phi}{\partial u}$ , $\frac{\partial \phi}{\partial v}$ which form simply the standard basis $(1,0), (0,1)$ at any point. We can use the (for now a bit alianating) notation $  \frac{\partial}{\partial u}$, $\frac{\partial}{\partial v}$ for the two bases vectors, dropping the $\phi$. We can the define a vector field
\[\mathcal V(u,v) = u \frac{\partial }{\partial v} - v \frac{\partial }{\partial u}.\]
Note that $\frac{\partial }{\partial v}$ really only is a strange notation for $\frac{\partial \phi}{\partial v} = (0,1)$ and the same for $\frac{\partial }{\partial u}$. We can now try to express the vector field $\mathcal V$ in a different map; in polar coordinates:
\[h(r,\omega) = (r \cos (\omega), r \sin (\omega))\]
The base of an arbitrary tangential space induced by this map is then
\[\frac{\partial}{\partial r} = (\cos(\omega), \sin(\omega)) \]
\[\frac{\partial}{\partial \omega} = (-r \sin(\omega), r\cos(\omega))\]
again using the fancy notation $\frac{\partial}{\partial r}$ to denote a vector. Expressed in euclidean coordinates given by $\phi$ these two vectors are
\[\frac{\partial}{\partial r} = \frac{1}{\abs{(u,v)}} ( u \frac{\partial }{\partial u} + v \frac{\partial }{\partial v} ) \]
\[\frac{\partial}{\partial \omega} = u \frac{\partial }{\partial v} - v \frac{\partial }{\partial u}\]
such that $\mathcal V$ expressed in polar coordinates is simply
\[\mathcal V = \frac{\partial}{\partial \omega}\]

\note{Image}

\section{Derivatives, Vectorfields and Differential Operators}
On manifolds you can calculate derivatives relative to a vector field. This is simply the directional derivative relative to the vector field's direction. Given a function $f$ we denote the derivative with respect to the vector field $\mathcal V$ as $\mathcal V (f)$. Geometrically we already did that, if $\alpha$ is a curve with $\alpha(0) = p$ and $\alpha'(0)= v = \mathcal V (p)$, then
\[\mathcal V (f) (p) = \frac{\partial }{\partial t} f \circ \alpha(t)\]
Now if $\mathcal V(u)$ is written in some map $\phi$ as $\sum_i v_i(u) \frac{\partial}{\partial u_i}$, again using the $\frac{\partial}{\partial u_i}$ as fancy notations of the induced base vectors the derivative with respect to the vector field becomes
\[\mathcal V (f) = \sum_i v_i(u)\frac{\partial(f \circ \phi)}{\partial u_i}\]
which motivates the 'strange' vector notation. \note{further reasoning needed?}

\section{Riemannian Metric}
From standard calculus we are used to that the gradient of a function $f:\mathbb R^k \rightarrow \mathbb R$ is a vector field with vectors pointing in the direction where $f$ has the largest increase. But in fact the gradient of a function $f$ is in fact only a linear mapping that approximates $f$ via $f(x + h) \approx f(x) + grad_x(f) (h)$ and the vector is merely a representation of the gradient. What actually happens here is that we represent the gradient using a vector AND a scalar product:
\[grad_x(f) (h) = \langle grad, h\rangle\]

To do the same on manifolds $M^k$ we need a scalar product on all tangential spaces. As we consider manifolds as objected embedded in a higher dimensional space $M^k \subset \mathbb R^n$ all tangential spaces are subspaces of the embedding space such that they inherit a scalar product. So if $\phi: \mathbb R^k \rightarrow M^k \subset \mathbb R^n$ is a local map inducing the local bases $\frac{\partial \phi}{\partial u_i}$, $i= 1...k$ to the tangential spaces and we have two vectors $v, w$ in some tangential space $T_p M^k$ expressed in the local bases as
\[v= v_1 \frac{\partial \phi}{\partial u_1} +...+ v_k\frac{\partial \phi}{\partial u_k} \]
\[w = w_1 \frac{\partial \phi}{\partial u_1} +...+ w_k\frac{\partial \phi}{\partial u_k}\]
the scalar product induced by the embedding space is
\[\langle v,w \rangle = \sum_{i,j = 1}^k v_iw_j\langle \frac{\partial \phi}{\partial u_i},\frac{\partial \phi}{\partial u_j}\rangle.\]
So if $v = (v_1,...,v_k)$ and $w = (w_1,...,w_k)$ in the map induced base, the induced scalar product is represented by the matrix
\[G= \begin{pmatrix}\langle \frac{\partial \phi}{\partial u_1},\frac{\partial \phi}{\partial u_1}\rangle &\cdots& \langle \frac{\partial \phi}{\partial u_1},\frac{\partial \phi}{\partial u_k}\rangle \\
\vdots &&\vdots\\
\langle \frac{\partial \phi}{\partial u_k},\frac{\partial \phi}{\partial u_1}\rangle &\cdots& \langle \frac{\partial \phi}{\partial u_k},\frac{\partial \phi}{\partial u_k}\rangle \end{pmatrix} = (D\phi)^T D\phi\]
This set of scalar products that is consistently defined for all tangential spaces $T_p M^k$ is the so called Riemannian metric.

Equipped with the Riemannian metric we can define a gradient and a gradient vector field for functions as well as divergience and a laplacian. \note{We will generalize these later but it is worth seeing them once for themselves before the generalization. TODO}

Note the Riemannian metric can also be used to measure volumes and angles. Angles are quite obvious; if we have two curves $\alpha$ and $\beta$ intersecting at some point $p$ with tangents $v$ and $w$, the angle between the curves is $\langle v,w\rangle$. The volume comes from the fact that $det(A A^T)$ equals the volume spanned by $A$'s row vectors squared. The determinant $det(G)$ then is the volume spanned by the column vectors of $D\phi$ squared and the $k$-dimensional volume of some subset $\phi(U) \subset M^k$ covered by a map $\phi$ is
\[vol(\phi(U))= \int_U \sqrt{det(G)}\;du_1,...,du_k\]
This measures 'absolute' orientation independent volume.



\section{Some Differential Operators}

The Goal of Discrete Differential Forms is to define discrete differential operators on Meshes, such that the retain some important geometric properties. 
You probably have seen Operators like divergence, gradients, rotation or the Laplace Operator. They arise very naturally in many settings. Most of the time you will have seen those operators written in standard Kartesian coordinates, like
\[\nabla = (\frac{ \partial}{\partial x_1},\frac{ \partial}{\partial x_2},\frac{ \partial}{\partial x_3})\]
where $x_1, x_2, x_3$ are the usual coordinates. If then $f$ is expressed in such coordinates, i.e. $f = f(x_1,x_2,x_3)$ it is obvious what $\nabla f$ should be. But what if $f$ is expressed in a different set of coordinates, e.g. $f(y_1,y_2,y_3)$? But for the coordinates $f$ would still describe the same function, so there should be an operation $\tilde{\nabla}$ that does the same to $f$ in these new coordinates as $\nabla$ did in the old coordinates.

Of course you can just formalize the change of coordinates and deduce what $\tilde{\nabla}$ is. We will try to give geometric coordinate free properties of these operators, which should enhance the understanding of them and which we will use to define discrete version of them, or that helps understand in what the discrete operators coincide with the continuous operators.

Differential forms unify these operators more from an algebraic than a geometric point of view. Thus a general Operator might have different (even if somehow related) geometric meanings, when applied to different objects.

\subsection{Divergence}

The divergence Operator is defined for vector valued functions. In Kartesian coordinates it is 
\[\nabla \cdot f = (\frac{ \partial}{\partial x_1},\frac{ \partial}{\partial x_2},\frac{ \partial}{\partial x_3})\cdot f = \frac{ \partial}{\partial x_1} f + \frac{ \partial}{\partial x_2}f + \frac{ \partial}{\partial x_3}f\]

But what does that mean? One way to look at divergence is with flows. Let $f$ be a vector field that describes the velocity (i.e. direction and speed) of a ''fluid'' at any position. For a closed Volume $V$ with surface $\partial V = S$ we then can calculate the netflow into the volume: how much flows out minus how much flows into this volume.

To determine this it is enough to look at the boundary of this volume and determine the flow through the boundary. This gives rise to this expression:

\[\int_{\partial V} f \cdot n ds\]
which measures the net flow, where $n$ are the surface normals at the given points. Now lets restrict $f$ to be a linear map $f(x) = A x$ with $A$ being a Matrix and $V$ be an axis aligned Volume with widths $h_1,h_2,h_3$ and base point $x_0,y_,z_0$, surface Areas $A_1, A_2, A_3$ and normals $n_1,n_2,n_3$. Then

\[\int_{\partial V} f \cdot n ds = \int_{\partial V} Ax \cdot n ds = \int_{\partial V} Ax \cdot n ds\]
\[= \int_{\partial A_x} A(x + h_xn_x) \cdot n_x ds - \int_{\partial A_x} A(x) \cdot n_x ds + ...\]
\[= \int_{\partial A_x} A(h_xn_x) \cdot n ds + ... = \sum_{i=1}^3 h_i n_i^TA n_i Area(A_i) = \sum_{i=1}^3 n_i^TA n_i Vol(V) = Tr(A) Vol(V)\]
So you see that the Trace of $A$ describes the netflow of a linear vector field. 

Back to the Divergence. Any function $f$ an locally be described with its Jacobi Matrix $Df$
\[Df = (\partial f_i / \partial f_j)_{i,j}\]
by 
\[f \approx f(x_0) + Df \cdot (x-x_0) \]
If we then look at local netflow at some point by letting shrink a Volume $V$ to the point  
\[\lim_{Vol(V) \rightarrow 0}\frac{\int_{\partial V} f \cdot n ds}{Vol(V)}\]
and approximate $f$ by its Jacobian, we get
\[\lim_{Vol(V) \rightarrow 0}\frac{\int_{\partial V} Df \cdot n ds}{Vol(V)} = Tr(Df)= \frac{ \partial}{\partial x_1} f + \frac{ \partial}{\partial x_2}f + \frac{ \partial}{\partial x_3}f\]
which actually is the divergence of $f$! This means the Divergence of a vectorvalued function $f$ is geometrically the local netflow.

\subsection{Gradient}
The gradient of a real valued function is fairly easy.

\subsection{Rotation}
Rotation is often denoted as $rot$ or $\nabla \times$ and is defined for vector valued functions and returns a vector valued function.  In Kartesian Coordinates this is
\[\nabla \times f = \left( \frac{ \partial f_3}{\partial x_2}- \frac{ \partial f_2}{\partial x_3}, \frac{ \partial f_1}{\partial x_3}- \frac{ \partial f_3}{\partial x_1}, \frac{ \partial f_2}{\partial x_1}- \frac{ \partial f_1}{\partial x_2}\right)\]
And what does this one do? The name already tells it; it measures the rotation of the vector field around some axis. One coordinate free way to describe it is
\[n\cdot rot(A) = \lim_{Area(F)\rightarrow 0 } \frac{\int_{\partial F} A \cdot dx}{Area(F)} \]
This maybe needs some explanation.... (image) That this extends the definition in cartesian coordinates directly follows from stokes Theorem, as the divergence Definition would have as well.


\subsection{Laplacian}
Deserves an own chapter.

Closedness and coclosedness of the laplace beltrami Operator. Riemann surfaces by Farkas and Kra..

The Laplacian is given by
\[\delta d + d \delta\]
and a form is said to be harmonic if
\[\delta d + d \delta f = 0\]
An important property is that $f$ is harmonic if and only if
\[df = 0\]
\[\delta f = 0\]
This is an extremely strong property and somewhat easier to understand geometrically. What is nice also is that it is very easy to show. We already know that
\[\langle \delta \omega,\phi\rangle = \langle \omega, d\phi\rangle\]
i.e. that $\delta$ and $d$ are adjoint. But then
\[\delta d + d \delta f = 0 \Rightarrow \langle \delta d + d \delta f, f\rangle = 0\]
and
\[0 = \langle \delta d + d \delta f, f\rangle = \langle \delta d f, f\rangle  + \langle d \delta f, f\rangle  = \langle d f, d f\rangle +\langle \delta f, \delta f\rangle\]
As the inner product of a form with itself is not negative both these terms have to be zero. And therefore
\[df = 0\]
\[\delta f = 0\]
But what does that mean geometrically? Well this depends on the form we are looking at. For Vector fields (1 Forms) in $\mathbb R^3$ this means they are divergence and rotation free. For a harmonic 2 Form (''Pseudo Vector Field'') this means.... for a 0 Form this means... constant???? wtf?
See e.g. %http://en.wikipedia.org/wiki/Hodge_dual#The_codifferential about the adjointness and http://en.wikipedia.org/wiki/De_Rham_cohomology about harmonic forms.
Fact is the adjointness is the root of my problem and HAS to be true only conditionally!. Or the innerproduct may be degenerated on open or infinite sets..???? I dont know

But consider an integral over Rn of a harmonic form ... its not defined. Over open Sets...? ...... domt lmpw. would expect it to be ok.
It might be that this adjointness only holds on compact or borderless manifolds or when 0 or constant on border. This here is a major hitch in my understanding...! 

Ahahahahaha! this is also mentioned on p 370 in my smart book (the yello one, you know which one.)
-As trace of the hesse, or derivative of the determinant of the ... or sth.